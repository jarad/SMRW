% !TeX root = SMRW.tex

\chapter{Statistical modeling}

The previous part introduce the concept of probability and a number of 
probability distributions.
Statistical modeling involves assuming a probability model for your data,
i.e. assuming a probability distribution for your data
where the parameters of the distribution are unknown.
Statistical inference involves making statements about those parameters.

Throughout this part (similar to the previous part),
we will utilize the notation where an upper case Roman letter, e.g. $Y$,
indicates your data before you observe it
while a lower case Roman letter, e.g. $y$,
indicates the observed value for your data. 
We now introduce Greek letters for the unknown parameters of the probability 
model for your data.

As an example, 
we may assume our data come from a binomial distribution.
Specifically, 
we assume that each individual trial has a common success probability and 
that each attempt is independent (given the true success probability) on 
all of the other attempts. 
Then 
\[ Y\sim Bin(n,\theta) \]
where 
\begin{itemize}
\item $Y$ is the total number of success,
\item n is the (known) number of attempts, and
\item $\theta$ is the unknown probability of success.
\end{itemize},
 
Another example is to assume that our data come from a normal distribution.
Specifically,
we assume that each observation has a common mean and variance and that each 
observation is independent of all other observations (given the true mean and
variance).
Then 
\[ Y_i\ind N(\mu,\sigma^2) \]
where 
\begin{itemize}
\item $Y_i$ is the measured observation for unit $i$ for $i=1,\ldots,n$,
\item $\mu$ is the unknown population mean, and
\item $\sigma^2$ ($\sigma$) is the unknown population variance (standard 
deviation).
\end{itemize}


\section{Population parameters}

The unknown parameters of these distributions are known as the {\bf population 
parameters}.
In $Y\sim Bin(n,\theta)$, $\theta$ is the  
population \emph{probability of success}.
In $Y_i\ind N(\mu,\sigma^2)$, $\mu$ and $\sigma^2$ ($\sigma$) are the 
population \emph{mean} and \emph{variance (standard deviation)}, respectively.
One way to think about these population parameters is that they are the actual
value in the entire population.
In the binomial example,
$\theta$ is the actual proportion of successes in the entire population.
In the normal example, 
$\mu$ and $\sigma^2$ are the actual mean and variance of the entire population.
Of course, we never observe the entire population.

\section{Sample statistics}

In order to make statements about these population parameters, 
we will collect a sample, 
i.e. a group of observations from the population.
In this sample, 
we can calculate statistics. 
So we might calculate the sample proportion of successes which is $y$ 
(the number of observed successes) divided by $n$.
