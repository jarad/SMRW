% !TeX root = SMRW.tex

\chapter{Frequentist inference}

Frequentist inference is based on considering what could have happened if 
different data sets had been obtained.
Thus from this perspective, 
the data are considered random.


\section{Confidence intervals and \pvalue{}s}

\subsection{Confidence interval}

For a scalar $\theta$,
a \emph{$100(1-a)$\% confidence interval} is a {\bf random} interval of the form 
$(L,U)$ such that the probability (among the different sets of data that could 
have been obtained) that the interval contains the true value is $100(1-a)$.

A \emph{(two-sided) equal-tailed $100(1-a)$\% confidence interval} has 
probability $a/2$ that $\theta<L$ and 
probability $a/2$ that $\theta>U$.
A \emph{one-sided lower $100(1-a)$\% confidence interval} has $L=-\infty$ 
(or whatever the lower bound for the support of $\theta$ is) and
probability $a$ that $\theta>U$
while an \emph{one-sided upper $100(1-a)$\% confidence interval} has $U=\infty$ 
(or whatever the upper bound for the support of $\theta$ is) and
probability $a$ that $\theta<L$.





\subsection{\pvalue{}}

For a given model $p_{y|\theta}(y|\theta_0)$ and a proposed parameter value 
$\theta_0$, a \emph{\pvalue{}} is the probability 
(when the data are considered random)
of observing a test statistic as or more extreme than that observed if the null
hypothesis is true.

To be completed...

\subsection{Relationship between confidence intervals and pvalues}

A two-sided $100(1-a)$\% confidence interval will contain $\theta_0$ 
if and only if the pvalue for the test with null hypothesis 
$H_0:\theta=\theta_0$ is greater than $a$.
A one-sided lower $100(1-a)$\% confidence interval will contain $\theta_0$
if and only if the pvalue for the test with null hypothesis 
$H_0:\theta\ge\theta_0$ is greater than $a$.
A one-sided upper $100(1-a)$\% confidence interval contains $\theta_0$
if and only if the pvalue for the test with null hypothesis 
$H_0:\theta\le\theta_0$ is greater than $a$.



\subsection{Asymptotic confidence intervals and \pvalue{}s}

To be completed...

\section{Binomial model}

Assume the binomial model $Y\sim Bin(n,\theta)$. 

\subsection{Confidence intervals}


\subsubsection{Based on CLT}

An approximate $100(1-a)$\% confidence interval can be constructed from the 
asymptotic distibution for $Y/n$ which, according to the Central Limit Theorem
is 
\[
\hat\theta \asim N\left(\theta, \frac{\theta(1-\theta)}{n} \right)
\]
where $\hat\theta = Y/n$.
Thus, an approximate two-sided $100(1-a)$\% confidence interval is
\[ 
\hat\theta \pm z_{a/2} \sqrt{\frac{\hat\theta(1-\hat\theta)}{n}}.
\]
An approximate one-sided lower $100(1-a)$\% confidence interval is
\[
\left[0,\hat\theta + z_{a} \sqrt{\frac{\hat\theta(1-\hat\theta)}{n}}\right).
\]
An approximate one-sided upper $100(1-a)$\% confidence interval is
\[
\left(\hat\theta - z_{a} \sqrt{\frac{\hat\theta(1-\hat\theta)}{n}},1\right].
\]

\subsubsection{Based on credible intervals}

Approximate $100(1-a)$\% confidence intervals can also be constructed by finding
the associated $100(1-a)$\% credible intervals described in Section 
\ref{s:binomial_credible_intervals}.


\subsection{\pvalue{}s}

Exact \pvalue{}s can be calculated for the binomial model, 
but if $n$ is large (and $y\ge 5$ and $n-y\ge 5$) 
approximate \pvalue{}s based on the CLT can be constructed.

\subsubsection{Exact \pvalue{}s}

The test statistic for exact \pvalue{}s is $Y\sim Bin(n,\theta_0)$ and
the observed value is $y$. 
For the two-sided test of $H_0: \theta=\theta_0$ vs $H_A: \theta\ne\theta_0$,
the exact pvalue is 
\[
P(|Y-n\theta_0|\ge |y-n\theta_0|).
\]

For the one-sided test of $H_0: \theta\le\theta_0$ vs $H_A: \theta>\theta_0$,
the exact pvalue is 
\[
P(Y\ge y).
\]

For the one-sided test of $H_0: \theta\ge\theta_0$ vs $H_A: \theta<\theta_0$,
the exact pvalue is 
\[
P(Y\le y).
\]

\subsubsection{Approximate \pvalue{}s}

To be completed ... (ignore below)

The test statistic for exact \pvalue{}s is $Y/n\asim N(\theta_0, $ and
the observed value is $y/n$. 
For the two-sided test of $H_0: \theta=\theta_0$ vs $H_A: \theta\ne\theta_0$,
the exact pvalue is 
\[
P(|Y-n\theta_0|\ge |y-n\theta_0|).
\]

For the one-sided test of $H_0: \theta\le\theta_0$ vs $H_A: \theta>\theta_0$,
the exact pvalue is 
\[
P(Y\ge y).
\]

For the one-sided test of $H_0: \theta\ge\theta_0$ vs $H_A: \theta<\theta_0$,
the exact pvalue is 
\[
P(Y\le y).
\]
